
\section{研究动机}

工欲善其事,必先利其器。在确定了选题之后,我们小组首先对本次课设的工具进行了分析,首先是不同网络仿真模拟器之间的区别。在研究中比较流行的几类仿真模拟器之中,NS3的编程语言支持Python,这对深度学习乃至强化学习的研究比较友好。

\begin{table}[h]
	\footnotesize
	\begin{center}
	\caption{不同模拟器之间的比较}
	\begin{tabular}{|l|l|l|l|l|l|l|l|} 
		\hline 
	 	& 可获得模型 & 网络层次模拟 & 仿真模型 & 编程语言 & GUI支持 & 是否开源\\ 
		\hline  
		OPNET & TCP/IP,ATM,Ethenet & 支持 & FSM & C++ & 是 & 否 \\ 
		\hline
		OMNeT++ & TCP/IP, SCSI ,FDDI & 支持 & FSM/Thread & C++ & 是 & 是 \\ 
		\hline 
		NS2 & 围绕TCP/IP & 不支持 & FSM & C++/Tcl & 否 & 是 \\
		\hline
		NS3 & & & FSM & C++/Python & & 是\\
		\hline
	\end{tabular}
	\end{center}
\end{table}

同时为了快速上手强化学习在计算机网络仿真之中的应用,我们小组将以自适应视频流传输作为本次课设的切入点,所以我们小组的课程设计将会由这两个方面构成。

