\section{总结}

\subsection{完成工作}

钟嘉伦:
自适应视频流传输调研,模型移植调研,各种文档撰写。

孙昊海:
RL方法调研,强化学习模型搭建,强化学习模型调优。

\subsection{个人小结}

钟嘉伦:
这次的计算机网络课设对我还是有很大帮助的。首先是了解了计算机网络仿真的相关知识,比如学会了NS3仿真的基本操作以及计算机网络应用的系统结构。其次是了解了强化学习的基本过程,特别是强化学习在计算机网络领域的具体应用,从算法研究走向实际的工程之中。此外,课程答辩的准备以及文档方面的撰写也与以往不同,我写了两个版本的报告,由于对Latex不太熟练且图比较多,Latex版直到最后一刻还在调图片的显示。所以十分感谢有这样一次机会能让我发现自己的不足,从而提升自己。

孙昊海:
这次实验对我的启发非常大,之前一直是做的深度学习的内容,涉及到的领域都大多是自然语言处理以及图像处理等方面,这是我第一次接触强化学习的内容,强化学习在计算机网络上的应用也是非常的广。
ns3的使用之前也不太熟悉,有很多不懂的地方,花了非常多的时间去翻看别人的代码。tcp拥塞控制是一个很棒的选题,强化学习在这个应用上会发挥非常大的作用,在后续过程当中,我想把现有的模型继续调优,达到一个比较不错的效果。